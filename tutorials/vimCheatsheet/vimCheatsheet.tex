\documentclass{article}
\usepackage[utf8]{inputenc}
\usepackage[english]{babel}
\usepackage{multicol}
\setlength{\columnsep}{1cm}
\usepackage{fancyhdr}
\usepackage{framed}
\usepackage{url}
\usepackage{anysize}
\marginsize{16pt}{14pt}{0pt}{40pt}
\pagestyle{fancy}
\renewcommand\headrule{}
\rhead{UCR-CS100FALL}


\begin{document}
\begin{center}
\Large\textbf{Vim Cheatsheet}\\
\textit{Itsuki David Fukada, Jesus Christian Reyes}
\end{center}

\begin{framed}
\begin{center}
\textbf{Format of command is: "Verb Range Motion"}
\end{center}
\begin{verbatim}
For example, if you use d2w, it will delete 2 words.(To exit insert or visual mode use <Esc>)
*If you do not use any range, it will automatically set it to 1. Also some verb can be used repeatedily.
*Also Certain words allow "i" to be a range which i stands for inclusion. For example,dw will delete 
the word from the cursor but diw will delete the entire word. 
\end{verbatim}
\end{framed}
\begin{multicols}{2}



\begin{center}
\textbf{Verb/Editing}
\end{center}
\begin{description}
\item[\ttfamily{gg}]
-Go to (20gg goes to line 20)
\item[\ttfamily{f+character}]
-Finds first instance of character
\item[\ttfamily{r}] 
-Replace a single character
\item[\ttfamily{J}] 
-Join the line below with the current
\item[\ttfamily{c}]
-Replace (combine with Motion and/or Range)
\item[\ttfamily{x}]
-Deletes the character at current cursor (cut)
\item[\ttfamily{d}]
-Delete (cut)
\item[\ttfamily{D}]
-Deletes from cursor to end of line
\item[\ttfamily{y}]
-Yank the highlighted text (copy)
\item[\ttfamily{p}]
-Paste the clipboard
\item[\ttfamily{u}]
-Undo
\item[\ttfamily{U}] 
-Returns the last line modified to its original state
\item[\ttfamily{Ctrl+r}]
-Redo
\item[\ttfamily{\textgreater}]
-Indent to the right
\item[\ttfamily{\textless}]
-Indent to the left
\item[\url{~}]
-Changes charater's case
\end{description}
\begin{center}
\textbf{Insertion/Deletion}
\end{center}
\begin{description}
\item[\ttfamily{i}]
-Insert mode at current cursor
\item[\ttfamily{I}] 
-Insert mode at beginning of line
\item[\ttfamily{a}]
-Insert mode after the cursor
\item[\ttfamily{A}]
-Insert mode at end of line.
\item[\ttfamily{o}]
-Append a new line below the current line
\item[\ttfamily{O}]
-Append a new line above the current line
\item[\ttfamily{s}]
-Delete word and enter insert mode
\item[\ttfamily{S}]
-Delete line and enter insert mode
\end{description}
\columnbreak

\begin{center}
\textbf{Motion/Navigation}
\end{center}
\begin{description}
\item[\ttfamily{h}] 
-Left
\item[\ttfamily{j}]
-Down
\item[\ttfamily{k}]
-Up
\item[\ttfamily{l}] 
-Right
\item[\ttfamily{w}]
-Jump forward to the start of a word
\item[\ttfamily{W}] 
-Jump forward to the start of a word (can contain punctuation)
\item[\ttfamily{e}]
-Jump forward to the end of a word
\item[\ttfamily{E}]
-Jump forward to the end of a word (can contain punctuation)
\item[\ttfamily{b}]
-Jump backward to the start of a word
\item[\ttfamily{B}] 
-Jump backward to the start of a word (can contain punctuation)
\item[\ttfamily{0 or} \^]
-Jump to the start of the line
\item[\ttfamily{\$}]
-Jump to the end of the line
\item[\ttfamily{\%}]
-Jump to matching bracket
\item[\ttfamily{G}]
-Go to end of file
\item[\ttfamily{H}]
-(High)Go to top of visible page
\item[\ttfamily{M}]
-(Middle)Go to middle of visible page
\item[\ttfamily{L}]
-(Low)Go to bottom of visible page
\item[\ttfamily{(}]
-Jump forward one sentence
\item[\ttfamily{)}]
-Jump backward one sentence
\item[\ttfamily{\{}]
-Jump forward one paragraph
\item[\ttfamily{\}}]
-Jump backward one paragraph
\item[\ttfamily{``(backticks)}]
-Return to the line where cursor was before latest jump
\end{description}

%\columnbreak
\begin{center}
\textbf{Other Navigations}
\end{center}
\begin{description}
\item[\ttfamily{Crtl-U}]
-Scroll Up
\item[\ttfamily{Ctrl-D}]
-Scroll down
\item[\ttfamily{Ctrl-E (Efter)}]
-Scroll down one line
\item[\ttfamily{Ctrl-Y (Yore)}]
-Scroll up one line
\item[\ttfamily{:NUMBER}]
-Go to line NUMBER
\item[\ttfamily{NUMBER + G}] 
-Go to line NUMBER (i.e.: 25G)
\end{description}
\end{multicols}



\newpage




\begin{center}
\Huge\textbf{Commands}
\end{center}

\begin{multicols}{3}
\begin{center}
\Large\textbf{Save/Exit}
\end{center}
\begin{description}
\item[:q] \hfill
-Quit
\item[:q!] \hfill
-Quit (Ignores changes made)
\item[:w] \hfill
-Save
\item[:wq (:x)] \hfill
-Save and quit
\item[:w FILENAME] \hfill
-Saves into FILENAME
\end{description}
\begin{center}
\Large\textbf{Tabs/Split}
\end{center}
\begin{description}
\item[\ttfamily{:tabe}]
-Creates a new tab
\item[\ttfamily{:tabe FILENAME}]
-Creates a new tab with name FILENAME
\item[\ttfamily{gt}] 
-Moves to next tab
\item[\ttfamily{gT}] 
-Moves to previous tab
\item[\ttfamily{\#gt}]
-Move to tab number \#
\end{description}

\columnbreak
\begin{center}
\Large\textbf{Substitution/Searching}
\end{center}
\begin{description}
\item[\ttfamily{:s/OLD/NEW/g}]
-Replaces all inteances of OLD in current line and replaces with NEW
\item[\ttfamily{:\%s/OLD/NEW/g}] 
-Replaces all instances of OLD with NEW
\item[\ttfamily{:\%s/OLD/NEW/gc}]
-Replaces all instances of old with new, but asks for confirmation
\end{description}
\begin{description}
\item[\ttfamily{/TEXT}] 
-Search for TEXT
\item[\ttfamily{?TEXT}] 
-Search for TEXT backwards
\item[\ttfamily{n}] 
-Move to next word that matches
\item[\ttfamily{N}]
-Move to previous word that matches
\end{description}

\begin{center}
\Large\textbf{Colorschemes}
\end{center}
\begin{description}
\item[\ttfamily{:color \textless TAB \textgreater}]
-Navigate through default colorschemes
\newline
*You can add colorschemes in \url{~/.vim/colors}
\\
\end{description}
\begin{center}
\Large\textbf{Other Commands}
\end{center}
\begin{description}
\item[\ttfamily{:!COMMAND}]
-Runs COMMAND (bash commands)
\item[\ttfamily{:r FILENAME}] 
-Copy FILENAME into your current file
\item[\ttfamily{:r!COMMAND}]
-Copy the output of COMMAND into current file
\end{description}

\begin{center}
\Large\textbf{Set Options}
\end{center}
\begin{description}
\item[\ttfamily{:set ic}]
-Set ignore case
\item[\ttfamily{:set autoindent}]
-Automatically indents
\item[\ttfamily{:set cindent}] 
-Automatically indents in C style
\item[\ttfamily{:set nu or :set number}] 
-Show line number.
\item[\ttfamily{:set incsearch}]
-Incremental search
\item[\ttfamily{:set hls}] 
-Highlighted search
\end{description}
\end{multicols}


\begin{multicols}{3}
\begin{center}
\Large\textbf{Visual Commands}
\end{center}
\begin{description}
\item[\ttfamily{Ctrl+v}]
-Enter visual mode	(highlight)
\item[\ttfamily{v}]
-Start visual mode from cursor 
\item[V] \hfill
-Start linewise visual mode
\item[\ttfamily{\textless}] 
-Shift text left
\item[\textgreater] \hfill
-Shift text right
\end{description}
\begin{center}
\Large\textbf{Macros}
\end{center}
\begin{description}
\item[\ttfamily{q+character}]
-Start recording to register character
\item[\ttfamily{q}]
-Stop recording
\item[\ttfamily{@+character}]
-Execute Macro on character
\item[\ttfamily{@@}]
-Execute the same macro again
\end{description}

\columnbreak

\begin{center}
\Large\textbf{Marks}
\end{center}
\begin{description}
\item[\ttfamily{m+LETTER}]
-Set mark LETTER at cursor location
\item[\ttfamily{'LETTER}]
-Jump to line of mark LETTER
\item[\ttfamily{`LETTER}]
-Jump to position mark LETTER
\item[\ttfamily{d'LETTER}]
-Delete from current line to line of mark LETTER
\item[\ttfamily{d`LETTER}]
-Delete from current cursor position to position of mark LETTER
\item[\ttfamily{c'a}]
-Change text from current line to line of mark LETTER
\item[\ttfamily{y`a}]
-Yank text to buffer from cursor to position mark LETTER
\end{description}

\columnbreak
\begin{center}
\Large\textbf{Buffers}
\end{center}
\begin{description}
\item[\ttfamily{:e file}]
-Edit existing or new file
\item[\ttfamily{:new}]
-Creates new window displaying contents of new buffer
\item[\ttfamily{:ls}]
-Lists all buffers
\item[\ttfamily{Ctrl-W w}]
-Move cursor to other window
\item[\ttfamily{:vsp}]
-Split window vertically
\item[\ttfamily{:bd}]
-Delete the current buffer. will fail if unsaved.
\end{description}
\end{multicols}


\begin{center}
\Large\textbf{Line Folding}
\end{center}

\begin{multicols}{3}
\begin{description}
\item[\ttfamily{"Highlighted text" + zf}] 
-Folds highlighted text
\item[\ttfamily{zf "NUMBER" j}]
-Creates folds from cursor down NUMBER lines
\item[\ttfamily{zf/STRING}]
-Creates fold from cursor to STRING
\item[\ttfamily{zj}] 
-Move to next fold
\item[\ttfamily{zk}]
-Move to previous fold
\item[\ttfamily{zo}]
-Open fold at cursor
\item[\ttfamily{ZO}]
-Open all folds
\item[\ttfamily{zm}]
-Increase fold level by one
\item[\ttfamily{zM}]
-Close all open folds
\item[\ttfamily{zr}]
-Decrease fold level by one
\item[\ttfamily{zR}]
-Decrease fold level to 0
\item[\ttfamily{zd}] 
-Delete fold at cursor
\item[\ttfamily{zD}]
-Delete all folds
\item[\ttfamily{\texttt{[}z}]
-Move to start of open fold
\item[\ttfamily{\texttt{]}z}]
-Move to end of open fold
\item[\ttfamily{:set=foldmethod=indent}]
-Automatically creats folds at every indent
\end{description}



\end{multicols}


\end{document}


