\documentclass{article}

\usepackage[margin=0.5in]{geometry}
\usepackage{listings}
\lstset{numbers=left,frame=single}

\title{Quiz 1 - shell commands and git}
\author{CS100 - software construction}

\begin{document}
\maketitle

\noindent
Answer the questions below based on the following bash session.

\begin{lstlisting}
$ mkdir quiz
$ cd quiz
$ git init
$ echo "This is an empty README file" > README
$ git add README
$ git commit -m "starting project"
$ git tag beginning
$ git branch coding
$ git checkout coding
$ echo "int main() { return 0; }" > main.cpp
$ g++ main.cpp
$ git add main.cpp
$ git commit -m "added some code"
$ ls 
$ touch LICENSE
$ git add LICENSE
$ git commit -m "added a license file"
$ git checkout beginning
$ ls -a
$ mkdir src
$ echo "int main() {return 1; }" > src/main.cpp
$ git add src/main.cpp
$ git commit -m "added different code"
$ git checkout coding
$ ls -a
\end{lstlisting}

\begin{enumerate}
\item On line 14, what does the \lstinline{ls} command display?
\vspace{0.5in}

\item On line 19, what does the \lstinline{ls -a} command display?
\vspace{0.5in}

\item On line 25, what does the \lstinline{ls -a} command display?
\vspace{0.5in}

\item Draw the tree that represents the git repository at the end of the session.
\vspace{0.5in}

\newpage
\item What does the \lstinline{touch} command do?
\vspace{0.5in}

\item What is the difference between \lstinline{cat} and \lstinline{echo}?
\vspace{0.5in}

\item If you want detailed information about what a unix command does, where should you look?  (Or, what command should you run?)
\vspace{0.5in}

\item In \lstinline{vim}, what is the command to move the cursor to the top of the page?
\vspace{0.5in}

\item In \lstinline{vim}, what is the command to move the cursor to the bottom of the page?
\vspace{0.5in}

\item In \lstinline{vim}, what is the command to delete the whole line the cursor is on?
\vspace{0.5in}

\item What Linux distribution do the lab machines have installed?
\vspace{0.5in}

\item Who invented the unix operating system?
\vspace{0.5in}

\end{enumerate}

\end{document}
