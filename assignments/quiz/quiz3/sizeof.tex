\documentclass{article}

\usepackage[margin=0.5in]{geometry}
\usepackage{listings}
\lstset{numbers=left,frame=single}

\title{Quiz 3 - sizeof}
\author{CS100 - software construction}

\newcommand{\sizeof}[1]{\texttt{sizeof({#1})~~~}}
\newcommand{\ptr}[1]{\sizeof{{#1}} & \sizeof{{#1}*} & \sizeof{{#1}**}\vspace{0.45in}\\\hline}
\newcommand{\ptrU}[1]{
    \ptr{{#1}}
    \ptr{unsigned {#1}}
    }

\begin{document}
\maketitle

On a 64 bit machine (e.g. one of the lab machines), what do the following applications of \texttt{sizeof} return?
\vspace{0.15in}

\noindent
\begin{tabular}{p{2.3in}p{2.3in}p{2.3in}}
\hline
\ptrU{char}
\ptrU{short}
\ptrU{int}
\ptrU{long}
\ptrU{long long}
\ptr{float}
\ptr{double}
\end{tabular}

\newpage
Given the following code:
\begin{lstlisting}
    struct s1 { char a; };
    struct s2 { char a; char b; };
    struct s3 { char a; int b; };
    struct s4 { char a; int b; char c; };
    struct s5 { char a; char b; int c; };
\end{lstlisting}
Fill out the table below in the same manner.
\vspace{0.15in}

\noindent
\begin{tabular}{p{2.3in}p{2.3in}p{2.3in}}
\hline
\ptr{s1}
\ptr{s2}
\ptr{s3}
\ptr{s4}
\ptr{s5}
\end{tabular}

\vspace{0.5in}
Write the output of each \lstinline{cout} statement in the following code:

\begin{lstlisting}
#include <iostream>
using namespace std;

int main()
{
    cout << (011 | 010) << endl;


    cout << (011 & 010) << endl;


    cout << (111 | 010) << endl;


    cout << (111 & 010) << endl;


    cout << 2 << 4 << endl;


    cout << (2 << 4) << endl;


    return 0;
}
\end{lstlisting}
\end{document}
