\documentclass{article}
\usepackage[utf8]{inputenc}
\usepackage[english]{babel}
\usepackage{multicol}
\setlength{\columnsep}{1cm}
\usepackage{fancyhdr}
\usepackage{framed}
\usepackage{url}
\usepackage{anysize}
\marginsize{14pt}{12pt}{0pt}{20pt}
\pagestyle{fancy}
\renewcommand\headrule{}
\rhead{UCR-CS100 SPRING 2015}

\begin{document}
\begin{center}
\LARGE\textbf{GDB Cheatsheet}\\
\textit{Dylan O'Neill}
\end{center}

\begin{framed}
\begin{center}
\Large\textbf{Format of commands in GDB: \ttfamily{Verb Argument}}
\end{center}
\begin{itemize}
\item Before running gdb, compile your source file with the \ttfamily{-g} \normalfont{flag} so gdb can debug our executable
\item GDB takes an executable as a command line argument and should be run by: \ttfamily{gdb ExecutableFile} \normalfont{.}
\item In gdb, many of the commands such as break are reduced to leeter like b and can be used intechangibly.
\item To exit gdb, type in \ttfamily{q} \normalfont{or} \ttfamily{quit}\normalfont{.} \\
\end{itemize}
\end{framed}
\begin{multicols}{2}
\begin{center}
\textbf{\LARGE{Break}}\\
\end{center}
\large\begin{tabular}{lll}
\ttfamily{b functionName} & Suspend the program at\\
\hspace{3ex}lineNumber & specified line number\\
\ttfamily{b +offset} & Sets breakpoint offset lines\\ 
\hspace{3ex}\ttfamily{-offset} & ahead/behind from last\\
& executed position\\
\ttfamily{b file:function} & Sets breakpoint at specified\\
\hspace{3ex}\ttfamily{file:lineNum} & line in the specified file\\
\ttfamily{b lineNum if condition} & Creates breakpoint if \\
& condition is met\\
\ttfamily{tbreak} & Breaks once then removed\\
\ttfamily{watch condition} & Suspends program when\\
& condition is met\\
\ttfamily{clear} & Clears breakpoint(s) at line\\
& number or inside function\\
\ttfamily{d} & Deletes all breakpoints\\
\ttfamily{c} & Continue to next breakpoint\\
\ttfamily{finish} & Continue to end of function\\
\end{tabular}
\\
\\
\\
\columnbreak

\begin{center}
\textbf{\LARGE{Line Execution}}
\end{center}
\setlength{\tabcolsep}{0.1cm}
\large\begin{tabular}{ll}
\ttfamily{s \#ofLines} & Step to next line of code\\
\ttfamily{n \#ofLines} & Execute next line of code\\
\ttfamily{si} & Step next assembly instruction\\
\ttfamily{ni} & Execute assembly instruction\\
\ttfamily{until lineNum} & Execute until lineNum.\\
\ttfamily{where} & Displays current line\\
& number and function\\
\ttfamily{r} & Start program execution at\\
& the beginning of the program\\
\ttfamily{c} & Continue executing until\\
& next breakpoint\\
\ttfamily{kill} & Stop program execution\\
\ttfamily{q} & Quits gdb\\

\end{tabular}

%Special Commands
\begin{center}
\textbf{\LARGE{Tracing}}
\end{center}
\setlength{\tabcolsep}{0.1cm}
\large\begin{tabular}{ll}
\ttfamily{bt} & Shows trace of where you are currently\\
\ttfamily{backtrace full} & Prints all local variables\\
\ttfamily{f} & Show current stack frame\\
\ttfamily{up} & Move up one frame\\
\ttfamily{down} & Move down one frame\\
\end{tabular}
\end{multicols}



\newpage




\begin{center}
\Large\textbf{Commands}
\end{center}

\setlength{\tabcolsep}{0.55cm}
\begin{multicols}{2}
\begin{center}
\large\textbf{Command Line Arguments}
\end{center}
\begin{tabular}{ll}
\ttfamily{--help} & list command\\
& line arguments\\
\ttfamily{-e ExecFile} & Identify executable\\
\ttfamily{-p PID} & Attaches to \\
& specified PID\\
\ttfamily{-cd=directory} & Specify current\\
& working directory\\
& when running gdb\\
\end{tabular}
\begin{center}
\large\textbf{Source}
\end{center}

\setlength{\tabcolsep}{0.2cm}
\begin{tabular}{ll}
\ttfamily{l line\#} & Lists source code\\
\hspace{3ex}function\\
\ttfamily{set listsize} & Number of lines listed\\
\ttfamily{dir} & Add directory to front\\
& of source code path\\
\end{tabular}

\begin{center}
\large\textbf{Variables/Memory}
\end{center}

\setlength{\tabcolsep}{0.7cm}
\begin{tabular}{ll}
\ttfamily{x 0x\textit{address}} & Examine contents\\
& of memory\\
\ttfamily{p var} & Prints value of var\\
\ttfamily{p *\textit{array}@\textit{length}} & Prints length number\\
& of elements of array\\
\ttfamily{p/x} & Prints int in hex\\
\ttfamily{p/d} & Prints int as signed int\\
\ttfamily{p/u} & Prints int as unsigned int\\
\ttfamily{p/t} & Prints int in binary\\
\ttfamily{p/c} & Prints int as char\\
\ttfamily{p/a} & Prints as hex address\\
\ttfamily{ptype} & Prints type def of variable\\
\end{tabular}
\\
\columnbreak
\begin{center}
\large\textbf{Modes}
\end{center}
\setlength{\tabcolsep}{0.1cm}
\begin{tabular}{ll}
\ttfamily{set logging} & Logs session, default\\
& file is \ttfamily{gdb.txt}\normalfont{}\\
\ttfamily{set logging file} & Sets file to store log\\
\ttfamily{set print array} & Readable format for arrays\\
\ttfamily{set print array-indexes} & Prints array indexes\\
\ttfamily{set print pretty} & Formart printing of\\
& C structures\\
\ttfamily{set print demangle} & Controls printing of\\
& C++ names\\
\end{tabular}
\\
\\
\\
\\
\\
\\
\\
\end{multicols}




\end{document}


